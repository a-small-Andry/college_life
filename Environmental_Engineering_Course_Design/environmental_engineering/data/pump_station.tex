\subsection{提升泵房}
提升泵泵房是污水处理厂中的一个重要设施,用于将污水从低处提升到高处,以便进行后续的处理和排放。


\subsubsection{水泵总扬程估算}
选泵前总扬程计算\cite[p.189]{《第2册排水工程》}:
\begin{equation}
	H=H_{ss}+H_{sd}+\sum{h_{s}}+\sum{h_{d}}
\end{equation}
式中:$H_{ss}$——吸水高度,为集水池内最低水位与水泵轴线之高差,m;
\newline\phantom{式中:}$H_{sd}$——压水高度,为泵轴线与输水最高点(即压水管出口处)之高差,m;
\newline\phantom{式中:}$\sum{h_{s}}$和$\sum{h_{d}}$——污水通过吸水管路和压水管路中的水头损失(包括沿程损失和局部损失),m。

根据《室外排水设计标准》的规定,污水泵设计扬程应根据设计流量时集水池水位与出水管渠水位差、水泵管路系统水头损失及安全水头三部分确定。
本项目设计参数如下表 \ref{tab:Lay out the design parameters} 所示:

\begin{table}[H]
	\centering
	\caption{布置设计参数}
	\begin{tabular}{cc}
	\toprule
	位置信息  & 标高(m) \\
	\midrule
	厂区设计地坪绝对标高 & 498.00 \\
	污水处理厂进水泵房处污水进水管管底标高 & 493.15 \\
	排水河流的最高水位 & 494.40 \\
	排水河流最低水位 & 491.80 \\
	常年平均水位 & 493.00 \\
	集水池池底标高 & 492.00 \\
	\bottomrule
	\end{tabular}%
	\label{tab:Lay out the design parameters}
\end{table}

\begin{enumerate}
	\item 集水池水位与出水管渠水位差 $H_{ss}+H_{sd} = \Delta H$
	
	% 出水管渠水位以及集水池水位的不同组合,可组成不同的扬程,设计流量时,出水管渠水位与集水池设计水位之差加上管路系统水头损失和安全水头为设计扬程;设计最小流量时,出水管渠水位与集水池设计最高水位之差加上管路系统水头损失和安全水头为最低工作扬程;	设计最大流量时,出水管渠水位与集水池设计最低水位之差 1 m 上管路系统水头损失和安全水头为最高工作扬程。安全水头一般为 $0.3 \sim 0.5$ m。
	
	设出水管管绝对标高 503.00 m,则
	\begin{align}
		H_{ss}+H_{sd} &= \Delta H \\
		&= 503.00-492.00-1.00 \;\text{m} \notag \\
		&= \eval{503-492-1} \;\text{m} \notag
	\end{align}

	\item 构筑物水头损失
	
	各构筑物水头损失见下表 \ref{tab:Loss of head of the structure} :
	\begin{table}[H]
		\centering
		\caption{构筑物水头损失}
		\begin{tabular}{cc}
		\toprule
		构筑物名称 & 水头损失(m) \\
		\midrule
		格栅    & 0.5 \\
		% 膜格栅   & 0.9 \\
		曝气沉砂池 & 0.5 \\
		氧化沟 & 0.6 \\
		二沉池   & 0.3 \\
		污泥浓缩池   & 0.9 \\
		消毒池   & 0.3 \\
		管路系统水头损失 & 1.0 \\
		\bottomrule
		\end{tabular}%
		\label{tab:Loss of head of the structure}
	\end{table}
	则可以求出各构筑物水头总损失 $\sum{h_{s}}+\sum{h_{d}} = \sum\limits_{i=1}^{n}\Delta P$:
	\begin{align}
		\sum{h_{s}}+\sum{h_{d}} &= \sum_{i=1}^{n}\Delta P \\
		&= 0.5+0.5+0.6+0.3+0.9+0.3+1.0 \;\text{m} \notag \\
		&= \eval{0.5+0.5+0.6+0.3+0.9+0.3+1.0} \;\text{m} \notag
	\end{align}

	\item 安全水头 $H_{\text{安全水头}}$
	
	由于污水泵站一般扬程较低,局部损失占总损失的比重较大,所以不可忽略。考虑到污水泵在使用过程中因效率下降和管道中阻力增加而增加的能量损失,在确定泵扬程时,可增大 $1 \sim 2$ m 安全扬程。则安全水头取 $H_{\text{安全水头}}=2$ m。
\end{enumerate}
故水泵总扬程估算为
\begin{align*}
	H &= H_{ss}+H_{sd}+\sum{h_{s}}+\sum{h_{d}} \\
	&= \Delta H + \sum\limits_{i=1}^{n}\Delta P + H_{\text{安全水头}} \\
	&= 10+4.1+2 \;\text{m} \\
	&= \eval{10+4.1+2} \;\text{m}
\end{align*}


\subsubsection{提升泵的选型}

因为设计最大流量 $Q_{max}=58100$ m$^3$/d,泵的个数为4\footnote{《室外排水设计标准》(GB 50014-2021):水泵台数不应少于2台,且不宜大于8台。},则每个泵的提升量为:
\begin{equation}
	Q_{\text{泵}}=\dfrac{Q_{max}}{4}=\dfrac{58100}{4}\;\text{m$^3$/d} =\eval{\dfrac{58100}{4}} \;\text{m$^3$/d}
\end{equation}

设水泵吸水管流速为 $v_0=1$ m/s \footnote{《室外排水设计标准》(GB 50014-2021):水泵吸水管设计流速宜为0.7 m/s $\sim$ 1.5 m/。},
出水管流速为 $v_1=2$ m/s \footnote{《室外排水设计标准》(GB 50014-2021):水泵出水管设计流速宜为0.8 m/s $\sim$ 2.5 m/s。}。
% \cite[p.39]{GB500142021}
根据《给水排水设计手册第 11 册常用设备(第三版)》,结合实际情况及计算所得的流量,选用 QXG 型潜水给水泵,其流量范围为 $50 \sim 3000$ m$^3$/h,扬程范围为 $5.5 \sim 65$ m,功率范围为 $7.5 \sim 250$ kW,可输送物理化学性质类似于水的液体,适用于城市、工厂、矿山、电站的给水排水和农田排涝灌溉等。QXG 型潜水给水泵类型如下表 \ref{tab:QXG type submersible feed pump} 所示:

\begin{table}[H]
	\centering
	\caption{QXG型潜水给水泵\cite{QXG型潜水给水泵}}
	\resizebox{\textwidth}{!}{
	\begin{tabular}{ccccccccccc}
	\toprule
	序号    & 规格型号  & 流量(Q)m$^3$/h & 扬程(H)m & 泵效率($\eta$)\% & 配套功率(P)kw & 转速(n)r/min & 口径(DN)$\phi$mm & 机座号   & 自耦装置型号 & 重量(W)kg \\
	\midrule
	1     & QXG300-5.5-7.5 & 300   & 5.5   & 70    & 7.5   & 980   & 150   & M160  & 150GAK & 280 \\
	2     & QXG50-32-11 & 50    & 32    & 75    & 11    & 1470  & 100   & M160  & 100GAK & 280 \\
	3     & QXG250-11-11 & 250   & 11    & 80.5  & 11    & 980   & 150   & M160  & 150GAK & 280 \\
	4     & QXG250-15-15 & 250   & 15    & 80.5  & 15    & 1470  & 150   & M160  & 150GAK & 280 \\
	5     & QXG400-9-15 & 400   & 9     & 80    & 15    & 1470  & 200   & M160  & 200GAK & 280 \\
	6     & QXG250-18-18.5 & 250   & 18    & 80    & 18.5  & 1470  & 150   & M180  & 150GAK & 450 \\
	7     & QXG400-11-18.5 & 400   & 11.5  & 80.5  & 18.5  & 1470  & 200   & M180  & 200GAK & 450 \\
	8     & QXG70-60-22 & 70    & 60    & 75    & 22    & 2900  & 100   & M180  & 100GAK & 500 \\
	9     & QXG140-35-22 & 140   & 35    & 80    & 22    & 1470  & 150   & M180  & 150GAK & 500 \\
	10    & QXG160-30-22 & 160   & 30    & 80    & 22    & 1470  & 150   & M180  & 150GAK & 500 \\
	11    & QXG250-21-22 & 250   & 21    & 79.5  & 22    & 1470  & 150   & M180  & 150GAK & 500 \\
	12    & QXG400-15-22 & 400   & 15    & 80    & 22    & 1470  & 200   & M180  & 200GAK & 500 \\
	13    & QXG600-9-22 & 600   & 9     & 82    & 22    & 980   & 250   & M200  & 250GAK & 600 \\
	14    & QXG250-27-30 & 250   & 27    & 76    & 30    & 980   & 200   & M225  & 200GAK & 800 \\
	15    & QXG400-18.5-30 & 400   & 18.5  & 80.5  & 30    & 980   & 200   & M225  & 200GAK & 800 \\
	16    & QXG600-12.5-30 & 600   & 12.5  & 82    & 30    & 980   & 250   & M225  & 250GAK & 800 \\
	17    & QXG900-8.5-30 & 900   & 8.5   & 83    & 30    & 980   & 250   & M225  & 250GAK & 800 \\
	18    & QXG250-33-37 & 250   & 33    & 75    & 37    & 1470  & 150   & M225  & 150GAK & 800 \\
	19    & QXG400-22-37 & 400   & 22    & 80    & 37    & 1470  & 200   & M225  & 200GAK & 800 \\
	20    & QXG600-15.5-37 & 600   & 15.5  & 80    & 37    & 1470  & 250   & M225  & 250GAK & 800 \\
	21    & QXG900-10-37 & 900   & 10    & 82    & 37    & 980   & 300   & M250  & 300GAK & 1200 \\
	22    & QXG1100-8-37 & 1100  & 8     & 82    & 37    & 980   & 300   & M250  & 300GAK & 1200 \\
	23    & QXG250-40-45 & 250   & 40    & 74    & 45    & 1470  & 200   & M225  & 200GAK & 1000 \\
	24    & QXG400-27-45 & 400   & 27    & 79    & 45    & 1470  & 200   & M225  & 200GAK & 800 \\
	25    & QXG600-18-45 & 600   & 18    & 80    & 45    & 1470  & 250   & M225  & 250GAK & 800 \\
	26    & QXG900-12.5-45 & 900   & 12.5  & 82    & 45    & 1470  & 300   & M225  & 300GAK & 800 \\
	27    & QXG1350-8.5-45 & 1350  & 8.5   & 83    & 45    & 740   & 400   & M280  & 400GAK & 1500 \\
	28    & QXG250-49-55 & 250   & 49    & 73    & 55    & 1470  & 200   & M250  & 200GAK & 1200 \\
	29    & QXG400-32.5-55 & 400   & 32.5  & 78    & 55    & 1470  & 200   & M250  & 200GAK & 1200 \\
	30    & QXG600-22-55 & 600   & 22    & 80    & 55    & 1470  & 250   & M250  & 250GAK & 1200 \\
	\bottomrule
	\end{tabular}}
	\label{tab:QXG type submersible feed pump}
\end{table}
\begin{table}[H]
	\centering
	\caption*{续表 \ref{tab:QXG type submersible feed pump} : QXG型潜水给水泵\cite{QXG型潜水给水泵}}
	\resizebox{\textwidth}{!}{
	\begin{tabular}{ccccccccccc}
	\toprule
	序号    & 规格型号  & 流量(Q)m$^3$/h & 扬程(H)m & 泵效率($\eta$)\% & 配套功率(P)kw & 转速(n)r/min & 口径(DN)$\phi$mm & 机座号   & 自耦装置型号 & 重量(W)kg \\
	\midrule
	31    & QXG900-15-55 & 900   & 15    & 82    & 55    & 1470  & 300   & M250  & 300GAK & 1500 \\
	32    & QXG1350-10-55 & 1350  & 10    & 83    & 55    & 980   & 400   & M280  & 400GAK & 1500 \\
	33    & QXG400-44-75 & 400   & 44    & 77.5  & 75    & 1470  & 200   & M280  & 200GAK & 1500 \\
	34    & QXG600-30-75 & 600   & 30    & 80    & 75    & 1470  & 250   & M280  & 250GAK & 1500 \\
	35    & QXG900-21-75 & 900   & 21    & 82    & 75    & 1470  & 300   & M280  & 300GAK & 1500 \\
	36    & QXG1350-14-75 & 1350  & 14    & 83    & 75    & 980   & 400   & M315  & 400GAK & 2300 \\
	37    & QXG2100-9-75 & 2100  & 9     & 84    & 75    & 590   & 500   & M315  & 500GAK & 2500 \\
	38    & QXG350-60-90 & 350   & 60    & 80    & 90    & 1470  & 200   & M280  & 200GAK & 1500 \\
	39    & QXG400-53-90 & 400   & 53    & 77    & 90    & 1470  & 200   & M280  & 200GAK & 1500 \\
	40    & QXG600-36-90 & 600   & 36    & 79    & 90    & 1470  & 250   & M280  & 250GAK & 1500 \\
	41    & QXG900-25-90 & 900   & 25    & 82    & 90    & 1470  & 300   & M280  & 300GAK & 1500 \\
	42    & QXG1350-17-90 & 1350  & 17    & 83    & 90    & 980   & 400   & M315  & 400GAK & 2300 \\
	43    & QXG2100-11-90 & 2100  & 11    & 84    & 90    & 740   & 500   & M315  & 500GAK & 2500 \\
	44    & QXG600-44-110 & 600   & 44    & 79    & 110   & 1470  & 250   & M315  & 250GAK & 2100 \\
	45    & QXG900-30-110 & 900   & 30    & 82    & 110   & 1470  & 250   & M315  & 250GAK & 2100 \\
	46    & QXG1350-20-110 & 1350  & 20    & 83    & 110   & 980   & 400   & M315  & 400GAK & 2300 \\
	47    & QXG2100-13-110 & 2100  & 13    & 84    & 110   & 980   & 400   & M315  & 400GAK & 2300 \\
	48    & QXG3000-9.5-110 & 3000  & 9.5   & 85    & 110   & 740   & 500   & M315  & 500GAK & 2500 \\
	49    & QXG400-65-132 & 400   & 65    & 75    & 132   & 1470  & 250   & M315  & 250GAK & 2100 \\
	50    & QXG600-52-132 & 600   & 52    & 78.5  & 132   & 1470  & 250   & M315  & 250GAK & 2100 \\
	51    & QXG900-35-132 & 900   & 35    & 81    & 132   & 1470  & 250   & M315  & 250GAK & 2100 \\
	52    & QXG1350-24-132 & 1350  & 24    & 83    & 132   & 980   & 400   & M315  & 400GAK & 2300 \\
	53    & QXG2100-16-132 & 2100  & 16    & 84    & 132   & 980   & 400   & M315  & 400GAK & 2300 \\
	54    & QXG3000-11-132 & 3000  & 11    & 85    & 132   & 740   & 500   & M315  & 500GAK & 2500 \\
	55    & QXG600-60-160 & 600   & 60    & 80.5  & 160   & 1470  & 250   & M315  & 250GAK & 2100 \\
	56    & QXG900-43-160 & 900   & 43    & 80.5  & 160   & 1470  & 300   & M315  & 300GAK & 2200 \\
	57    & QXG1350-30-160 & 1350  & 30    & 83    & 160   & 1470  & 400   & M315  & 400GAK & 2300 \\
	58    & QXG2100-19-160 & 2100  & 19    & 84    & 160   & 980   & 500   & M315  & 500GAK & 2500 \\
	59    & QXG3000-14-160 & 3000  & 14    & 85    & 160   & 740   & 500   & M355  & 500GAK & 3800 \\
	60    & QXG600-65-185 & 600   & 65    & 80    & 185   & 1470  & 250   & M355  & 250GAK & 3500 \\
	61    & QXG900-50-185 & 900   & 50    & 80    & 185   & 980   & 300   & M355  & 300GAK & 3750 \\
	62    & QXG1350-34-185 & 1350  & 34    & 82.5  & 185   & 980   & 400   & M355  & 400GAK & 3950 \\
	63    & QXG2100-22-185 & 2100  & 22    & 84    & 185   & 980   & 500   & M355  & 500GAK & 3950 \\
	64    & QXG3000-16-185 & 3000  & 16    & 85    & 185   & 980   & 500   & M355  & 500GAK & 3950 \\
	65    & QXG900-54-200 & 900   & 54    & 82    & 200   & 980   & 300   & M355  & 300GAK & 3950 \\
	66    & QXG1350-37-200 & 1350  & 37    & 82    & 200   & 980   & 400   & M355  & 400GAK & 3950 \\
	67    & QXG2100-24-200 & 2100  & 24    & 84    & 200   & 980   & 500   & M355  & 500GAK & 3950 \\
	68    & QXG3000-17-200 & 3000  & 17    & 85    & 200   & 980   & 500   & M355  & 500GAK & 3950 \\
	69    & QXG900-59-220 & 900   & 59    & 82    & 220   & 980   & 300   & M355  & 300GAK & 4100 \\
	70    & QXG1350-41-220 & 1350  & 41    & 82    & 220   & 980   & 400   & M355  & 400GAK & 4100 \\
	71    & QXG2100-27-220 & 2100  & 27    & 84    & 220   & 980   & 500   & M355  & 500GAK & 4100 \\
	72    & QXG3000-19-220 & 3000  & 19    & 85    & 220   & 980   & 500   & M355  & 500GAK & 4100 \\
	73    & QXG2100-30-250 & 2100  & 30    & 83.5  & 250   & 980   & 500   & M355  & 500GAK & 4250 \\
	74    & QXG3000-22-250 & 3000  & 22    & 85    & 250   & 980   & 500   & M355  & 500GAK & 4250 \\
	\bottomrule
	\end{tabular}}
\end{table}

根据设计需求(流量:$Q'_{\text{泵}}>Q_{\text{泵}}=14525/24 \;\text{m$^3$/h}\approx 606 \;\text{m$^3$/h}$;扬程:$H'>H=16.1 \;\text{m} \approx 17 \;\text{m}$),再结合上表 \ref{tab:QXG type submersible feed pump},选用 5 台 QXG900-21-75 型潜水给水泵,四用一备。

\begin{table}[H]
	\centering
	\caption{QXG900-21-75 型潜水给水泵性能参数}
	\resizebox{\textwidth}{!}{
	\begin{tabular}{ccccccccccc}
	\toprule
	序号    & 规格型号  & 流量(Q)m$^3$/h & 扬程(H)m & 泵效率($\eta$)\% & 配套功率(P)kw & 转速(n)r/min & 口径(DN)$\phi$mm & 机座号   & 自耦装置型号 & 重量(W)kg \\
	\midrule
	35    & QXG900-21-75 & 900 & 21 & 82    & 75    & 1470  & 300   & M280  & 300GAK & 1500 \\
	\bottomrule
	\end{tabular}}
	\label{tab:QXG900-21-75 submersible feed pump performance parameters}
\end{table}


\subsubsection{水泵最大安装高度}

泵的允许几何安装高度与多方面条件有关\cite{水泵最大安装高度如何计算},公式如下:
\begin{equation}
	\left[H_g\right]=\dfrac{p_e}{\rho g}-\dfrac{p_v}{\rho g}-\left[NPSH\right]-h_w
\end{equation}
式中:$\left[H_g\right]$——泵的允许几何安装高度,m;(计算结果供设计时利用,实际安装高度需低于允许安装高度)
\newline\phantom{式中:}$p_e$——吸水水面压力,Pa;(为吸水水面的大气压,海拔越高大气压越低)
\newline\phantom{式中:}$p_v$——饱和蒸汽压力,Pa;(与水温有关,水温越高,饱和蒸汽压力越高)
\newline\phantom{式中:}$\rho$——流体的密度,kg/m$^3$;
\newline\phantom{式中:}$g$——重力加速度,9.81 m/s$^2$;
\newline\phantom{式中:}$\left[NPSH\right]$——水泵的允许汽蚀余量,m;(与水泵性能有关,由水泵厂家提供)
\newline\phantom{式中:}$h_w$——吸入管路中的水头损失,m。

\begin{table}[H]
  \centering
  \caption{不同海拔时的大气及对应的水头高度\cite{水泵最大安装高度如何计算}}
    \begin{tabular}{ccc}
    \toprule
    海振高度(m) & 大气压力(kPa) & 水头高度(m) \\
    \midrule
    -600  & 110.85 & 11.3 \\
    0     & 101.32 & 10.3 \\
    200   & 99.08 & 10.1 \\
    500   & 95.16 & 9.7 \\
    100   & 90.25 & 9.2 \\
    1500  & 84.36 & 8.6 \\
    2000  & 79.46 & 8.1 \\
    3000  & 70.63 & 7.2 \\
    4000  & 61.8  & 6.3 \\
    5000  & 53.95 & 5.5 \\
    \bottomrule
    \end{tabular}
	\label{tab:The atmosphere at different altitudes and the corresponding head height}
\end{table}%

\begin{table}[H]
  \centering
  \caption{不同温度时水的饱和蒸汽压对应水头高度\cite{水泵最大安装高度如何计算}}
    \begin{tabular}{ccc}
    \toprule
    水温(℃) & 饱和蒸汽压力(kPa) & 水头高度(m) \\
    \midrule
    10    & 1.23  & 0.125 \\
    20    & 2.34  & 0.238 \\
    30    & 4.24  & 0.433 \\
    40    & 7.37  & 0.752 \\
    50    & 12.33 & 1.272 \\
    60    & 19.92 & 2.066 \\
    70    & 31.16 & 3.249 \\
    80    & 47.36 & 4.97 \\
    90    & 70.1  & 7.406 \\
    100   & 101.32 & 10.786 \\
    \bottomrule
    \end{tabular}
	\label{tab:The saturated vapor pressure of water at different temperatures corresponds to the height of the head}
\end{table}%

因为该城镇污水处理厂的海拔高度为:$490 \sim 520$ m,吸水水面压力为当地大气压,,由表 \ref{tab:The atmosphere at different altitudes and the corresponding head height} 查得海拔 500 m处大气压力
\[\dfrac{p_e}{\rho g} = 9.7 \;\text{m}\]
气温:最冷月平均为 5℃,最热月平均为 32.5℃,最冷月水温在 $8 \sim 12$ 摄氏度取。由表 \ref{tab:The saturated vapor pressure of water at different temperatures corresponds to the height of the head} 估算出水的饱和蒸汽压头为
$$\dfrac{p_v}{\rho g} = \dfrac{1.272-0.752}{40-30}\times (32.5-30)+0.752 \;\text{m} = \eval{(1.272-0.752)/(40-30)\times (32.5-30)+0.752} \;\text{m}$$
QXG900-21-75 水泵汽蚀余量为 $$\text{[NPSH]r}=3.29 \;\text{m}$$
欲在海拔 500 m高度的地方工作,该地区夏季最高水温为32.5℃,取吸水管的水头损失为 $$h_w = 1 \;\text{m}$$
则该泵在当地的运行几何安装高度 [Hg] 计算如下:
\begin{align*}
	\left[H_g\right] &=\dfrac{p_e}{\rho g}-\dfrac{p_v}{\rho g}-\left[NPSH\right]-h_w \\
	&= 9.7-0.882-3.29-1 \;\text{m} \\
	&= \eval{9.7-0.882-3.29-1} \;\text{m}
\end{align*}


\subsubsection{集水池的计算}
污水泵站集水池的容积与进入泵站的流量变化情况、泵的型号、台数及其工作制度、泵站操作性质、启动时间等有关。同时,集水池的容积在满足安装格栅和吸水管的要求、保证泵工作时的水力条件以及能够及时将流人的污水抽走的条件下,应尽量小些。因为缩小集水池的容积,不仅能降低泵站的造价,还可减轻集水池污水中大量杂物的沉积和腐化。此外,污泥泵房集水池的容积应按一次排入的污泥量和污泥泵抽送能力计算确定。活性污泥泵房集水池的容积,应按排入的回流污泥量、剩余污泥量和污泥泵抽送能力计算确定。最后,在集水池中设置冲洗装置和清泥设施\cite{GB500142021}。

取$t_{min}=6$ min\footnote{《室外排水设计标准》(GB 50014-2021):污水泵站集水池的容积不应小于最大一台水泵 5 min的出水量。},则集水池的容量为
\begin{align}
	W = Q_{\text{泵}}\cdot t_{min}=\dfrac{14525}{24\times 60} \times 6 \;\text{m$^3$} =\eval{\dfrac{14525}{24\times 60} \times 6}[3] \;\text{m$^3$}
\end{align}

因为栅后槽总高度 $H=1.016$ m,栅前水深 $h=0.63$ m,进水渠宽 $B_1=1.26$ m,又因为“设计最高水位 = 进水管充满高度 + 水位保留高度”,则设置进水管充满高度为 0.63 m ,水位保留高度(泵的安装高度 $[H_g]=4.528$ m)为 4 m,取集水坑坑深为 600 mm\footnote{《室外排水设计标准》(GB 50014-2021):集水池池底应设置集水坑,坑深宜为 500 mm $\sim$ 700 mm。}。所以,可计算设计最高水位为
\begin{align}
	\text{设计最高水位} &= \text{进水管充满高度} + \text{水位保留高度} \\
	&= 0.63 + 4 \;\text{m} \notag \\
	&= \eval{0.63+4} \;\text{m} \notag
\end{align}


% \subsubsection{泵房布置}
% 泵房布置水泵房布置宜符合以下要求:
% \begin{enumerate}
% 	\item 水泵房的平面布置水泵布置宜采用单行布置,主要机组的布置和通道宽度,应满足机电设备安装、运行和操作的要求,即水泵机组基础间的净距不宜小于1.0m ,机组突出部分与墙壁的净距不宜小于1.2m,主要通道宽度不宜小于1.5m;配电箱前面的通道宽度,低压配电时不宜小于1.5m,高压配电时不宜小于 2.0m;当采用在配电箱后检修时,配电箱后距墙的净距不宜小于1.0m;有电动起重机的泵房内,应有吊装设备的通道。
% 	\item 水泵房的高程布置泵房各层层高应根据水泵机组、电气设备、起吊装置、安装、运行和检修等因素确定。水泵机组基座应按水泵的要求设置,并应高出地坪 0.lm 以上;泵房内地面敷设管道时,应根据需要设置跨越设施,若架空敷设时,不得跨越电气设备和阻碍通道,通行处的管底距地面不宜小于 2.0m;当泵房为多层时,楼板应设置吊物孔,其位置应在起吊设备的工作范围内,吊物孔尺寸应按所需吊装的最大部件外形尺寸每边放 0.2m 以上。
% \end{enumerate}

% 泵站室外地坪标高应按城镇防洪标准确定,并符合规划部门要求。泵房室内地坪应比室外地坪高 $0.2 \sim 0.3$ m 。易受洪水淹没地区的泵站,其人口处设计地面标高应比设计洪水位高 0.5m 以上,当不能满足上述要求时,可在人口处设置闸槽墩临时性防洪措施。
