\null\par
\section{实验一\hspace{1em}水环境质量现状评价}
\subsection{教学目的与要求}

\noindent\textbf{教学目的:}通过几组水环境质量参数现状数据实例来进行水环境质量现状评价。
\newline\textbf{教学要求:}掌握标准指数法和内梅罗水质指数在水环境质量评价中的应用。


\subsection{实验报告背景}
\subsubsection[水质指数法]{水质指数法\protect\cite{HJ2.3-2018}}

\paragraph*{一般性水质因子(随着浓度增加而水质变差的水质因子)的指数计算公式:}
\begin{equation} \label{eq:General water quality factors}
    S_{i,j}=\dfrac{C_{i,j}}{C_{si}}
\end{equation}
式中:
$S_{i,j}$——评价因子 $i$ 的水质指数,大于1表明该水质因子超标;
\newline\phantom{式中:}$C_{i,j}$——评价因子 $i$ 在 $j$ 点的实测统计代表值,mg/L;
\newline\phantom{式中:}$C_{si}$——评价因子 $i$ 的水质评价标准限值,mg/L。

\paragraph*{溶解氧(DO)的标准指数计算公式:}
\begin{equation} \label{eq:DO}
    S_{\mathrm{DO},j} = 
    \begin{cases}
        \dfrac{\mathrm{DO_s}}{\mathrm{DO}_j} &\qquad \mathrm{DO}_j \leqslant \mathrm{DO_f} \\
        & \\
        \dfrac{\left\lvert \mathrm{DO_f}-\mathrm{DO}_j \right\rvert}{\mathrm{DO_f}-\mathrm{DO_s}} &\qquad \mathrm{DO}_j > \mathrm{DO_f}
    \end{cases}
\end{equation}
式中:
$S_{DO,j}$——溶解氧的标准指数,大于1表明该水质因子超标;
\newline\phantom{式中:}$\mathrm{DO}_j$——溶解氧在 $j$ 点的实测统计代表值,mg/L;
\newline\phantom{式中:}$\mathrm{DO_s}$——溶解氧的水质评价标准限值,mg/L;
\newline\phantom{式中:}$\mathrm{DO_f}$——饱和溶解氧浓度,mg/L,对于河流,$\mathrm{DO_f}=468/(31.6+T)$,对于
\newline\phantom{式中:$\mathrm{DO_f}$——}盐度比较高的湖泊、水库及入海河口、近岸海域,
\newline\phantom{式中:$\mathrm{DO_f}$——}$\mathrm{DO_f}=(491-2.65S)/(33.5+T)$;
\newline\phantom{式中:}$S$——实用盐度符号,量纲一;
\newline\phantom{式中:}$T$——水温,℃。

\paragraph*{pH值的指数计算公式:}
\begin{equation} \label{eq:pH}
    S_{\mathrm{pH},j} = 
    \begin{cases}
        \dfrac{7.0-\mathrm{pH}_j}{7.0-\mathrm{pH_{sd}}} &\qquad \mathrm{pH}_j \leqslant 7.0 \\
        & \\
        \dfrac{\mathrm{pH}_j-7.0}{\mathrm{pH_{su}}-7.0} &\qquad \mathrm{pH}_j > 7.0
    \end{cases}
\end{equation}
式中:
$S_{\mathrm{pH},j}$——pH值的指数,大于1表明该水质因子超标;
\newline\phantom{式中:}$\mathrm{pH}_j$——pH值实测统计代表值;
\newline\phantom{式中:}$\mathrm{pH_{sd}}$——评价标准中pH值的下限值;
\newline\phantom{式中:}$\mathrm{pH_{su}}$——评价标准中pH值的上限值。

\paragraph*{内梅罗水质指数计算公式:}
\begin{equation} \label{eq:Nemero Water Quality Index}
    P_{\text{传统}} = \sqrt{\dfrac{F_{\text{最大}}^2+F_{\text{平均}}^2}{2}}
\end{equation}
式中:
$P_{\text{传统}}$——传统内梅罗污染指数;
\newline\phantom{式中:}$F_{\text{最大}}$——$F_i$的最大值;
\newline\phantom{式中:}$F_{\text{平均}}$——$F_i$的平均值。


\subsubsection{计算标准要求}

\begin{enumerate}
    \item 使用单项指数法(标准指数法)和内梅罗水质指数评价某项目环评监测点的水质。
    \item 评价标准为《地表水环境质量标准》(GB3838-2002)中的Ⅲ类标准,各标准值见下表 \ref{tab:Environmental quality standards for surface water}。
\end{enumerate}

\begin{table}[H]
    \centering
    \caption{地表水环境质量标准\cite{GB3838-2002}}
    \resizebox{\textwidth}{!}{
    \begin{tabular}{ccccccccccc}
        \multicolumn{11}{r}{单位:mg/L,pH为无量纲,水温20℃} \\
        \toprule
        pH    & $\mathrm{COD_{Mn}}$ & $\mathrm{BOD_5}$  & DO    & SS    & 氨氮    & 总磷    & 总氮    & 石油类   & LAS   & 动植物油 \\
        \midrule
        $6\sim 9$   & $\leqslant 6$    & $\leqslant4$    & $\geqslant 5$    & $\leqslant100$  & $\leqslant1.0$  & $\leqslant0.05$ & $\leqslant1.0$  & $\leqslant0.05$ & $\leqslant0.2$  & $\leqslant0.05$ \\
        \bottomrule
    \end{tabular}}
    \label{tab:Environmental quality standards for surface water}%
\end{table}


\subsection{报告主体内容}
\subsubsection{水质情况}
\begin{table}[H]
    \centering
    \caption{某项目环评水质监测点结果}
    \begin{tabular}{cccccc}
        \multicolumn{6}{r}{单位:mg/L,pH为无量纲} \\
        \toprule
        序号    & 污染物   & 红路水库北 & 红路水库南 & 石场水坑  & 流溪河灌渠 \\
        \midrule
        1     & pH    & 7.7   & 7.97  & 7.9   & 7.5 \\
        2     & DO    & 7.0     & 6.8   & 7.3   & 6.1 \\
        3     & $\mathrm{BOD_5}$  & 2$^{\mathrm{L}}$    & 2$^{\mathrm{L}}$     & 2$^{\mathrm{L}}$    & 3.05 \\
        4     & $\mathrm{COD_{Mn}}$ & 5.3   & 5.3   & 1.8   & 3.23 \\
        5     & 氨氮    & 0.26  & 0.24  & 0.05  & 0.27 \\
        6     & SS    & 54.0    & 48.0    & 28.0    & 37.0 \\
        7     & 总磷    & 0.04  & 0.11  & 0.10   & 0.18 \\
        8     & 总氮    & 1.30   & 1.06  & 1.03  & 1.22 \\
        9     & 石油类   & 0.2   & 0.21  & 0.02$^{\mathrm{L}}$ & 0.02$^{\mathrm{L}}$ \\
        10    & LAS   & 0.14  & 0.13  & 0.05$^{\mathrm{L}}$ & 0.17 \\
        11    & 动植物油  & 0.37  & 0.35  & 0.02$^{\mathrm{L}}$ & 0.45 \\
        \bottomrule
        \multicolumn{6}{l}{备注:低于最低检测浓度,结果以最低检出浓度右上角加L表示。} \\
    \end{tabular}%
    \label{tab:Results of water quality monitoring points for environmental impact assessment of a project}%
\end{table}%


\subsubsection{Excel处理步骤}
\begin{enumerate}
    \item 数据整合
    \begin{enumerate}[label=(\arabic*)]
        \item 根据《地表水环境质量标准》(GB3838-2002)中的 Ⅲ 类标准(见表 \ref{tab:Environmental quality standards for surface water})和某项目环评水质监测点结果(见表 \ref{tab:Results of water quality monitoring points for environmental impact assessment of a project})将它们穿插排列:
        \begin{itemize}
            \item 上行为水质监测点结果,下行为地表水环境质量标准;
            \item 留出空位填写计算后相应的水质指数,并用横线将不同类别的污染物分开。
        \end{itemize}
        \item 再根据公式 \ref{eq:Nemero Water Quality Index} 的计算要求,在Excel末端插入三行空白栏位(分别为最大值、平均值和$P_{\text{传统}}$);
        \item 最后添加适当的分割线作为修饰,并绘制于 $\mathrm{A1:F26}$ 区域(版式效果可见表 \ref{tab:Evaluation of water quality parameters})。
    \end{enumerate}

    \item pH 值的指数计算
    \begin{enumerate}[label=(\arabic*)]
        \item 根据公式 \ref{eq:pH} 计算要求,获取到评价标准中:
        \begin{itemize}
            \item pH 值的下限值 $\mathrm{pH_{sd}}=6$
            \item pH 值的上限值 $\mathrm{pH_{su}}=9$
        \end{itemize}
        \item 然后通过IF函数对每个计算的值进行判断,以相对应的计算公式输出结果:
        \begin{align*}
            \mathrm{=IF(C\$2<=7,(7-C\$2)/(7-6),(C\$2-7)/(9-7))}
        \end{align*}
        \item 最后在pH栏位中依次向右自动填充。
    \end{enumerate}
    
    \item 溶解氧(DO)的标准指数计算
    \begin{enumerate}[label=(\arabic*)]
        \item 根据公式 \ref{eq:DO} 计算要求,又因为该项目环评处于河流段,且水温为20℃,则饱和溶解氧浓度为:
        $$\mathrm{DO_f}=468/(31.6+T)=468/(31.6+20) \;\text{mg/L}$$ 
        \item 通过IF函数对每个计算的值进行判断,以相对应的计算公式输出结果:
        \begin{align*}
            = & \mathrm{IF(C\$4<=(468/(31.6+20)),5/C\$4,} \\
            & \mathrm{ABS((468/(31.6+20))-C\$4)/((468/(31.6+20))-5))}
        \end{align*}
        \item 最后在DO栏位中依次向右自动填充。
    \end{enumerate}
    
    \item 一般性水质因子的指数计算
    
    相较于pH 值的指数计算和溶解氧(DO)的标准指数计算,一般性水质因子的指数计算就显得非常简单,根据公式 \ref{eq:General water quality factors} 计算要求,输入计算公式后自动填充即可。

    \item 创建Excel规则
    
    水质因子超标的要求是水质指数大于1,单单是自己一个一个比对的话,会非常吃力,但如果在Excel中建立相关规则就可以事半功倍。具体操作如下:
    \begin{enumerate}[label=(\arabic*)]
        \item 选中所有水质指数的计算单元格,点击“条件格式”中的“新建规则”;
        \item 选择“使用公式确定要设置格式的单元格”并在“为符合此公式的值设置格式”中输入“$\mathrm{=C2>1}$”(C2为所有选中单元格的首个单元格);
        \item 将格式样式设置为单元格背景填充为红色。
    \end{enumerate}
    通过以上操作即可完成规则的创建,也就得到了表 \ref{tab:Evaluation of water quality parameters} 展示的效果。

    \item 内梅罗水质指数$P_{\text{传统}}$的计算
    
    根据公式 \ref{eq:Nemero Water Quality Index} 要求,通过MAXIFS函数得到$F_{\text{最大}}$,AVERAGEIF函数得到$F_{\text{平均}}$,最后由计算公式得到 $P_{\text{传统}}$。以pH的操作举例:
    \begin{enumerate}[label=(\arabic*)]
        \item $F_{\text{最大}}$:$\mathrm{=MAXIFS(C\$1:C\$23,\$A\$1:\$A\$23,\text{"标准"})}$
        \item $F_{\text{平均}}$:$\mathrm{=AVERAGEIF(\$A\$1:\$A\$23,\text{"标准"},C\$1:C\$23)}$
        \item $P_{\text{传统}}$:$\mathrm{=((C\$24\text{\textasciicircum} 2+C\$25\text{\textasciicircum} 2)/2)\text{\textasciicircum} 0.5}$
        \item 最后全部依次向右自动填充。
    \end{enumerate}
\end{enumerate}

经过以上Excel的处理步骤,可以得到完整的水质参数评价结果,如下表 \ref{tab:Evaluation of water quality parameters} 所示。


\subsubsection{水质参数评价结果与分析}
% \begin{table}[H]
%     \centering
%     \caption{水质参数评价结果}
%     \resizebox{\textwidth}{!}{
%     \begin{tabular}{cc|cccc|cc|c}
%         \specialrule{1pt}{0pt}{0pt} % 1pt是线宽
%         序号    & 污染物   & 红路水库北 & 红路水库南 & 石场水坑  & 流溪河灌渠 & 最大值   & 平均值   & $P_{\text{传统}}$ \\
%         \hline
%         1     & pH    & 7.70  & 7.97  & 7.90  & 7.5   &       &       &  \\
%         标准    & $6\sim 9$   & 0.350  & 0.485  & 0.450  & 0.250  & 0.485  & 0.3838  & 0.4373  \\
%         \hline
%         2     & DO    & 7.0   & 6.8   & 7.3   & 6.1   &       &       &  \\
%         标准    & $\geqslant 5$    & 0.714  & 0.735  & 0.685  & 0.820  & 0.820  & 0.7385  & 0.7802  \\
%         \hline
%         3     & BOD$_5$  & 2     & 2     & 2     & 3.05  &       &       &  \\
%         标准    & $\leqslant 4$    & 0.500  & 0.500  & 0.500  & 0.763  & 0.763  & 0.5656  & 0.6713  \\
%         \hline
%         4     & $\mathrm{COD_{Mn}}$ & 5.3   & 5.3   & 1.8   & 3.23  &       &       &  \\
%         标准    & $\leqslant 6$    & 0.883  & 0.883  & 0.300  & 0.538  & 0.883  & 0.6513  & 0.7760  \\
%         \hline
%         5     & 氨氮    & 0.26  & 0.24  & 0.05  & 0.27  &       &       &  \\
%         标准    & $\leqslant 1.0$  & 0.260  & 0.240  & 0.050  & 0.270  & 0.270  & 0.2050  & 0.2397  \\
%         \hline
%         6     & SS    & 54.0  & 48.0  & 28.0  & 37.0  &       &       &  \\
%         标准    & $\leqslant 100$  & 0.540  & 0.480  & 0.280  & 0.370  & 0.540  & 0.4175  & 0.4827  \\
%         \hline
%         7     & 总磷    & 0.04  & 0.11  & 0.10  & 0.18  &       &       &  \\
%         标准    & $\leqslant 0.05$ & 0.800  & \cellcolor[rgb]{ 1,  0,  0}2.200  & \cellcolor[rgb]{ 1,  0,  0}2.000  & \cellcolor[rgb]{ 1,  0,  0}3.600  & 3.600  & 2.1500  & \cellcolor[rgb]{ 1,  0,  0}2.9650  \\
%         \hline
%         8     & 总氮    & 1.30  & 1.06  & 1.03  & 1.22  &       &       &  \\
%         标准    & $\leqslant 1.0$  & \cellcolor[rgb]{ 1,  0,  0}1.300  & \cellcolor[rgb]{ 1,  0,  0}1.060  & \cellcolor[rgb]{ 1,  0,  0}1.030  & \cellcolor[rgb]{ 1,  0,  0}1.220  & 1.300  & 1.1525  & \cellcolor[rgb]{ 1,  0,  0}1.2285  \\
%         \hline
%         9     & 石油类   & 0.20  & 0.21  & 0.02  & 0.02  &       &       &  \\
%         标准    & $\leqslant 0.05$ & \cellcolor[rgb]{ 1,  0,  0}4.000  & \cellcolor[rgb]{ 1,  0,  0}4.200  & 0.400  & 0.400  & 4.200  & 2.2500  & \cellcolor[rgb]{ 1,  0,  0}3.3692  \\
%         \hline
%         10    & LAS   & 0.14  & 0.13  & 0.05  & 0.17  &       &       &  \\
%         标准    & $\leqslant 0.2$  & 0.700  & 0.650  & 0.250  & 0.850  & 0.850  & 0.6125  & 0.7408  \\
%         \hline
%         11    & 动植物油  & 0.37  & 0.35  & 0.02  & 0.45  &       &       &  \\
%         标准    & $\leqslant 0.05$ & \cellcolor[rgb]{ 1,  0,  0}7.400  & \cellcolor[rgb]{ 1,  0,  0}7.000  & 0.400  & \cellcolor[rgb]{ 1,  0,  0}9.000  & 9.000  & 5.9500  & \cellcolor[rgb]{ 1,  0,  0}7.6290  \\
%         \specialrule{1pt}{0pt}{0pt} % 1pt是线宽
%     \end{tabular}}
%     \label{tab:Evaluation of water quality parameters}%
% \end{table}%

\begin{table}[H]
    \centering
    \caption{水质参数评价结果}
    \begin{tabular}{cc|cccc}
        \specialrule{1pt}{0pt}{0pt} % 1pt是线宽
        序号 & 污染物 & 红路水库北 & 红路水库南 & 石场水坑 & 流溪河灌渠 \\
        \hline
        1 & pH & 7.70 & 7.97 & 7.90 & 7.5 \\
        标准 & $6\sim 9$ & 0.350 & 0.485 & 0.450 & 0.250 \\
        \hline
        2 & DO & 7.0 & 6.8 & 7.3 & 6.1 \\
        标准 & $\geqslant 5$ & 0.714 & 0.735 & 0.685 & 0.820 \\
        \hline
        3 & BOD$5$ & 2 & 2 & 2 & 3.05 \\
        标准 & $\leqslant 4$ & 0.500 & 0.500 & 0.500 & 0.763 \\
        \hline
        4 & $\mathrm{COD{Mn}}$ & 5.3 & 5.3 & 1.8 & 3.23 \\
        标准 & $\leqslant 6$ & 0.883 & 0.883 & 0.300 & 0.538 \\
        \hline
        5 & 氨氮 & 0.26 & 0.24 & 0.05 & 0.27 \\
        标准 & $\leqslant 1.0$ & 0.260 & 0.240 & 0.050 & 0.270 \\
        \hline
        6 & SS & 54.0 & 48.0 & 28.0 & 37.0 \\
        标准 & $\leqslant 100$ & 0.540 & 0.480 & 0.280 & 0.370 \\
        \hline
        7 & 总磷 & 0.04 & 0.11 & 0.10 & 0.18 \\
        标准 & $\leqslant 0.05$ & 0.800 & \cellcolor[rgb]{ 1, 0, 0}2.200 & \cellcolor[rgb]{ 1, 0, 0}2.000 & \cellcolor[rgb]{ 1, 0, 0}3.600 \\
        \hline
        8 & 总氮 & 1.30 & 1.06 & 1.03 & 1.22 \\
        标准 & $\leqslant 1.0$ & \cellcolor[rgb]{ 1, 0, 0}1.300 & \cellcolor[rgb]{ 1, 0, 0}1.060 & \cellcolor[rgb]{ 1, 0, 0}1.030 & \cellcolor[rgb]{ 1, 0, 0}1.220 \\
        \hline
        9 & 石油类 & 0.20 & 0.21 & 0.02 & 0.02 \\
        标准 & $\leqslant 0.05$ & \cellcolor[rgb]{ 1, 0, 0}4.000 & \cellcolor[rgb]{ 1, 0, 0}4.200 & 0.400 & 0.400 \\
        \hline
        10 & LAS & 0.14 & 0.13 & 0.05 & 0.17 \\
        标准 & $\leqslant 0.2$ & 0.700 & 0.650 & 0.250 & 0.850 \\
        \hline
        11 & 动植物油 & 0.37 & 0.35 & 0.02 & 0.45 \\
        标准 & $\leqslant 0.05$ & \cellcolor[rgb]{ 1, 0, 0}7.400 & \cellcolor[rgb]{ 1, 0, 0}7.000 & 0.400 & \cellcolor[rgb]{ 1, 0, 0}9.000 \\
        \hline
        \multicolumn{2}{c|}{最大值} & 7.400  & 7.000  & 2.000  & 9.000  \\
        \multicolumn{2}{c|}{平均值} & 1.5861  & 1.6758  & 0.5768  & 1.6437  \\
        \hline
        \multicolumn{2}{c|}{P传统} & 5.3514  & 5.0896  & 1.4719  & 6.4692  \\
        \specialrule{1pt}{0pt}{0pt} % 1pt是线宽
    \end{tabular}
    \label{tab:Evaluation of water quality parameters}%
\end{table}


根据表格中的计算数据,可以得到以下信息:
\begin{enumerate}
    \item pH值、溶解氧(DO)、五日生化需氧量(BOD$_5$)、高锰酸盐指数($\mathrm{COD_{Mn}}$)、氨氮、悬浮物(SS)和线性烷基苯磺酸盐(LAS)的浓度均满足标准要求。
    \item 红路水库南和红路水库北的总磷和总氮超过了标准值,表明水库的富营养化程度较高,需要采取相应的措施进行治理。
    \item 石场水坑、流溪河灌渠、红路水库北的石油类和动植物油超过了标准值,说明这些地点存在油类污染物的排放,需要加强监管和减少排放。
    \item 内梅罗水质指数值$P_{\text{传统}}$越大,表示水质状况越差;反之,如果内梅罗水质指数值$P_{\text{传统}}$较小,表示水质状况较好。从计算出的 $P_{\text{传统}}$ 可以看出,除了石场水坑,红路水库北、红路水库南和流溪河灌渠的$P_{\text{传统}}$都比较大,水质污染情况较为严重,需要关注和加强相关处理,减少其主要污染物的排放或产生。
\end{enumerate}

在水质污染评价中,我们还可以根据内梅罗水质指数污染程度的不同来划分等级标准,以判断水质的污染程度,并进行相应的评价和管理:
\begin{table}[H]
    \centering
    \caption{内梅罗水质指数污染等级划分标准}
    \begin{tabular}{cccccc}
    \toprule
    P     & <1    & $1\sim 2$   & $2\sim 3$   & $3\sim 5$   & >5 \\
    \midrule
    水质等级  & 清洁    & 轻污染   & 污染    & 重污染   & 严重污染 \\
    \bottomrule
    \end{tabular}
    \label{tab:Nemero Water Quality Index pollution classification standard}
\end{table}

根据内梅罗水质指数污染等级划分标准,我们可以进一步将区域污染程度进行可视化的等级评价:
\begin{table}[H]
    \centering
    \caption{区域污染等级评价}
      \begin{tabular}{ccc}
      \toprule
      区域    & $P_{\text{传统}}$   & 水质等级 \\
      \midrule
      红路水库北 & 5.3514  & 严重污染 \\
      红路水库南 & 5.0896  & 严重污染 \\
      石场水坑  & 1.4719  & 轻污染 \\
      流溪河灌渠 & 6.4692  & 严重污染 \\
      \bottomrule
      \end{tabular}%
    \label{tab:Regional pollution level evaluation}%
\end{table}%



\subsection{小结及建议}

在采取措施改善水质的时候,需要关注水质因子指数较高的监测点,特别是对红路水库南、红路水库北和流溪河灌渠的治理更为紧迫。以下是一些建议和意见:
\begin{enumerate}
    \item 总磷和总氮的超标可能与农业、污水排放等因素有关。建议加强农田和养殖场的管理,合理使用化肥和农药,并加强污水处理,以降低排放对水体的影响。
    \item 石油类和动植物油的超标可能与工业废水排放、油污等因素有关。建议严格控制工业废水的排放,加强油污治理和监管,确保工业和生活污水得到有效处理,避免对水体造成污染。
    \item 需要建立有效的监测和预警机制,及时发现和应对水质问题。定期对水质进行监测,并进行综合评估和分析,以便及时采取措施进行治理和改善。
    \item 提高公众的环保意识,加强环境教育,鼓励居民、企业和机构积极参与水质保护和治理工作,共同建设美丽的水环境。
\end{enumerate}

水质问题是一个复杂的系统工程,需要多方合作和持续努力才能取得明显的改善效果。希望相关部门和社会各界共同努力,保护和改善水环境,为人民群众提供更好的生活条件。

