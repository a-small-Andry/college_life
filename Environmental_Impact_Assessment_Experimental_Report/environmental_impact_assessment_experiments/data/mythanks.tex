%----------------------------------------------致谢----------------------------------------------%
\null   %--------只是为了空行

\begin{mythanks}

在完成本实验报告的过程中,我获得了许多宝贵的学习经验和技能,我在此向相关人员表示衷心的感谢。

首先,我要感谢我的指导教师,在整个实验过程中给予了我悉心的指导和支持。他们不仅传授了我水环境质量评价、大气环境质量评价、土壤环境质量调查和声环境质量评价的方法和要求,还引导我熟练运用标准指数法、内梅罗水质指数、AERSCREEN预测软件、EIAProA2018软件模型、EIAN2.0和Surfer等软件工具。指导教师的耐心指导和专业知识使我受益匪浅。

此外,我还要感谢实验报告中所涉及的各种软件工具,如\LaTeX、Excel和Python。借助\LaTeX,我能够高效地排版和编辑实验报告,使其具备专业的外观和结构。Excel在数据处理和分析方面发挥了重要作用,帮助我整理和统计实验数据。而Python作为一种强大的编程语言,为我提供了便捷的数据处理和可视化工具,使实验结果更加清晰和易于理解。

最后,我还要感谢实验中的同学们,他们与我合作完成了数据采集、分析和讨论。他们的合作精神和共同努力使我们能够顺利完成实验报告,共同取得了丰硕的成果。

总的来说,通过这次实验,我学会了水环境质量评价、大气环境质量评价、土壤环境质量调查和声环境质量评价的方法和要求,并掌握了标准指数法、内梅罗水质指数、AERSCREEN预测软件、EIAProA2018软件模型、EIAN2.0、Surfer和奥维互动地图软件(OMap)等软件的使用。再次感谢所有给予我支持和帮助的人员和工具。

致以最诚挚的谢意!

\vfill
\noindent\dotfill

\noindent 校园邮箱:
\href{mailto:andeli@stu.cdut.edu.cn}{andeli@stu.cdut.edu.cn}

\noindent \LaTeX 源数据(.tex)文件下载:
\href{https://github.com/a-small-Andry/college_life/tree/main/Environmental_Impact_Assessment_Experimental_Report}{https://github.com/a-small-Andry/college_life/tree/main}

\end{mythanks}
%----------------------------------------------致谢----------------------------------------------%