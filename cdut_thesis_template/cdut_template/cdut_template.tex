% !TeX encoding = UTF8
% !TeX program = xelatex
\documentclass[oneside]{cdut_thesis} %-- 格式可选参数有oneside(默认)和twoside

% 封面个人信息的修改
% 在 cdut_thesis.cls 中修改 \cdut(该报告的类型) 参数
\newcommand\yourtitle{成都理工大学环境科学与工程专业\LaTeX 报告模板}
\newcommand\yourname{安德利}
\newcommand\yourstudentnumber{202019020220}
\newcommand\yourteacher{XXX}
\newcommand\yourmajor{环境科学与工程}
\newcommand\yourcollege{成都理工大学生态环境学院环境科学与工程系}
\newcommand\yourfinishdate{2023年6月}

% 设置pdf的属性参数,书签已自动生成(见.cls)
\hypersetup{
	pdfauthor={\yourname},
	pdftitle={\yourtitle},
	% colorlinks=true, % 控制超链接的颜色,默认false为黑色
		% linkcolor=blue,
		% citecolor=green,
		% urlcolor=red
}

% \usepackage{style/Genshinbox}       % 自定义宏包 - 原神任务框 \Genshinsection{<标题名称>}
% \usepackage{style/pythonlistings}   % 自定义宏包 - 显示python代码 \begin{lstlisting}[caption={my file.py}, label={lst:python_code_my_file}] \end{lstlisting}

% \usepackage{pythontex} % 内部运行 python 代码,不会用就不用管
%%%%%%%%%%%%%%%%%%%%%%%%%%%%%%%%% 导言区配置 %%%%%%%%%%%%%%%%%%%%%%%%%%%%%%%%%


\begin{document}
%----- 封面 -----%
\thispagestyle{empty}
\begin{center}
    \makebox[109mm][c]{\heiti\zihao{-2} \cdut}              % \cdut 在类文件中定义为“成都理工大学普通本科实习报告”
\vskip2cm
    \makebox[109mm][c]{\heiti\zihao{2} \yourtitle}                     %报告的名称
\end{center}
\vskip2cm
\begin{center}
    \includegraphics[width=50mm,height=50mm]{school_badge/CDUT.png}                    %成都理工大学的校徽
\end{center}
\vskip2.5cm
%-------------------------------------------------个人信息---------------------------------------------------%
\begin{center}
    \makebox[30mm][l]{\heiti\zihao{3} 学 \hspace{0.8cm} 生:} \makebox[50mm][l]{\heiti\zihao{3} \yourname }
    %姓名
    \vskip0.5cm
    \makebox[30mm][l]{\heiti\zihao{3} 学 \hspace{0.8cm} 号:} \makebox[50mm][l]{\heiti\zihao{3} \yourstudentnumber }
    %学号
    \vskip0.5cm
    \makebox[30mm][l]{\heiti\zihao{3} 指导教师:} \makebox[50mm][l]{\heiti\zihao{3} \yourteacher }
    %指导教师
    \vskip0.5cm
    \makebox[30mm][l]{\heiti\zihao{3} 专 \hspace{0.8cm} 业:} \makebox[50mm][l]{\heiti\zihao{3} \yourmajor }
    %专业
\end{center}
%-------------------------------------------------个人信息---------------------------------------------------%
\vskip2.5cm
\begin{center}
    \makebox[109mm][c]{\heiti\zihao{-2} \yourcollege }    
\end{center}
\vskip1cm
\vfill\begin{center}
    {\heiti\zihao{3} \yourfinishdate }                                                   %撰写时间
\end{center}
\newpage
%-----------------------------空白页-----------------------------%
\makeatletter % 激活 @ 符号
\ifcdut@twoside % 文档为双面样式,则添加空白页
	\thispagestyle{empty}
	\null
	\newpage
\fi
\makeatother % 取消激活 @ 符号

%%%---------------------------------------(个人信息)要用时才做修改----------------------------------------%%%
%-----------------------------------------------------------------------------------------------------------%
%------------------------------------------------下划线格式--------------------------------------------------%
%\begin{center}
%    \makebox[30mm][l]{\heiti\zihao{3} 学 \hspace{0.8cm} 生:} \underline{\makebox[60mm][l]{\heiti\zihao{3} 安德利}}
%    %姓名
%    \vskip0.5cm
%    \makebox[30mm][l]{\heiti\zihao{3} 学 \hspace{0.8cm} 号:} \underline{\makebox[60mm][l]{\heiti\zihao{3} %202019020220}}
%    %学号
%    \vskip0.5cm
%    \makebox[30mm][l]{\heiti\zihao{3} 指导教师:} \underline{\makebox[60mm][l]{\heiti\zihao{3} 杜海英,吴灿,何鹏,徐婷婷}}
%    %指导教师
%    \vskip0.5cm
%    \makebox[30mm][l]{\heiti\zihao{3} 专 \hspace{0.8cm} 业:} \underline{\makebox[60mm][l]{\heiti\zihao{3} 环境科学与工程}}
%    %专业
%\end{center}
%------------------------------------------------下划线格式--------------------------------------------------%
%-----------------------------------------------------------------------------------------------------------%   %-- 导入封面

%----- 摘要 -----%
\setcounter{page}{1}   %-- 设置起始页脚
\pagenumbering{Roman}  %-- 设置页码为大写罗马数字

\null\par
\begin{chineseabstract}

这里写中文摘要……





















    
\par\null\par\null\par % 空两行
\noindent \textbf{\heiti 关键词:} 关键词一,关键词二    %  (3-5个关键词)
\end{chineseabstract}  %-- 中文摘要
% \null\par
\begin{englishabstract}
    
English……
















\par\null\par\null\par % 空两行
\noindent \textbf{Key words:} one, two    %  (3-5个关键词)
\end{englishabstract}  %-- 英文摘要

%----- 目录 -----%
{\thispagestyle{fancy}}  %-- 设置目录的页眉页脚
\null \tableofcontents         %--生成目录
{\thispagestyle{fancy}}  %-- 设置目录的页眉页脚

% \newpage\null\par \listoffigures {\thispagestyle{fancy}}  %-- 插图目录(可选)
% \newpage\null\par \listoftables {\thispagestyle{fancy}}   %-- 表格目录(可选)

\newpage
%----------------------------------------------正文----------------------------------------------%
\setcounter{page}{1}    %-- 设置起始页脚
\pagenumbering{arabic}  %-- 设置页码为阿拉伯数字
%-- 以上部分属于正文设置 --%

\null\par
\section{一级标题}
\subsection{二级标题}
\subsubsection{三级标题}
\paragraph{段落标题}~{}\par

正文部分……

\paragraph{普通段落标题}段落后面的文字


正文部分……


\subsection{尝试参考文献的引用}
phdthesis(博士学位论文)\cite{phdthesis_example}

article(期刊文章)\cite{article_example}

online(在线资源)\cite{online_example}

standard(标准文件)\cite{standard_example}

book(书籍)\cite{book_example}

inproceedings(会议论文)\cite{inproceedings_example}

report(报告)\cite{report_example}

thesis(学位论文)\cite{thesis_example}

misc(其他类型)\cite{misc_example}

如果为 [?] ,不用担心,是正常的,只需要通过 BibTeX  编译即可。



%----------------------------------------------正文----------------------------------------------%
%--------------------------------------结论、致谢、参考文献---------------------------------------%
%----------------------------------------------结论----------------------------------------------%
\null   %--------只是为了空行

\begin{conclusion}

这里写结论……

\end{conclusion}
%----------------------------------------------结论----------------------------------------------% % 结论

% 参考文献
\null\par
\phantomsection
\bibliographystyle{unsrt}
\bibliography{reference}

%----------------------------------------------致谢----------------------------------------------%
\null   %--------只是为了空行

\begin{mythanks}

在完成本实验报告的过程中,我获得了许多宝贵的学习经验和技能,我在此向相关人员表示衷心的感谢。

首先,我要感谢我的指导教师,在整个实验过程中给予了我悉心的指导和支持。他们不仅传授了我水环境质量评价、大气环境质量评价、土壤环境质量调查和声环境质量评价的方法和要求,还引导我熟练运用标准指数法、内梅罗水质指数、AERSCREEN预测软件、EIAProA2018软件模型、EIAN2.0和Surfer等软件工具。指导教师的耐心指导和专业知识使我受益匪浅。

此外,我还要感谢实验报告中所涉及的各种软件工具,如\LaTeX、Excel和Python。借助\LaTeX,我能够高效地排版和编辑实验报告,使其具备专业的外观和结构。Excel在数据处理和分析方面发挥了重要作用,帮助我整理和统计实验数据。而Python作为一种强大的编程语言,为我提供了便捷的数据处理和可视化工具,使实验结果更加清晰和易于理解。

最后,我还要感谢实验中的同学们,他们与我合作完成了数据采集、分析和讨论。他们的合作精神和共同努力使我们能够顺利完成实验报告,共同取得了丰硕的成果。

总的来说,通过这次实验,我学会了水环境质量评价、大气环境质量评价、土壤环境质量调查和声环境质量评价的方法和要求,并掌握了标准指数法、内梅罗水质指数、AERSCREEN预测软件、EIAProA2018软件模型、EIAN2.0、Surfer和奥维互动地图软件(OMap)等软件的使用。再次感谢所有给予我支持和帮助的人员和工具。

致以最诚挚的谢意!

\vfill
\noindent\dotfill

\noindent 校园邮箱:
\href{mailto:andeli@stu.cdut.edu.cn}{andeli@stu.cdut.edu.cn}

\noindent \LaTeX 源数据(.tex)文件下载:
\href{https://github.com/a-small-Andry/college_life/tree/main/Environmental_Impact_Assessment_Experimental_Report}{https://github.com/a-small-Andry/college_life/tree/main}

\end{mythanks}
%----------------------------------------------致谢----------------------------------------------% % 致谢
%--------------------------------------结论、致谢、参考文献---------------------------------------%

\end{document}
